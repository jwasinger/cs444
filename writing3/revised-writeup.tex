Memory management is an incredibly important for an operating system.  In order to make sure that memory is used efficiently, the kernel dolls out memory resources.  A common architectural theme among Windows, FreeBSD and Linux is virtual address spaces.  Virtual address spaces allow multiple processes to share a large physical address space by mapping a continuous virtual address spaces to non-contiguous portions of a larger physical address space.  On systems with multiple processes vying for system resources, virtual address spaces allow the details of physical memory to be abstracted away from the processes that are using the memory.

For all Windows, FreeBSD and Linux, Pages are the basic unit of memory allocation.

The kernel represents all physical pages on the system using the \textbf{struct page} structure.
Reference tracking is used to mark which pages are in use.  Pages in Linux represent the physical memory of a system (not virtual).  Thus, the contents of kernel pages are susceptible to change.

\begin{lstlisting}
struct page {
	unsigned long flags;
	atomic_t _count;
	atomic_t _mapcount;
	unsigned long private;
	struct address_space *mapping;
	pgoff_t index;
	struct list_head lru;
	void *virtual;
};
\end{lstlisting}

Many Linux systems support multiple page sizes.  32-bit architectures have 4KB pages while 64-bit architectures have 8KB pages (Love, 231).

Pages are further segregated into zones.  Zones better delineate the different properties of pages.  For example, some zones may refer to areas of high memory.  Other zones represent physical memory capable of performing Direct Memory Access.

\begin{lstlisting}
struct zone {
        spinlock_t              lock;
        unsigned long           free_pages;
        unsigned long           pages_min;
        unsigned long           pages_low;
        unsigned long           pages_high;
        unsigned long           protection[MAX_NR_ZONES];
        spinlock_t              lru_lock;
        struct list_head        active_list;
        struct list_head        inactive_list;
        unsigned long           nr_scan_active;
        unsigned long           nr_scan_inactive;
        unsigned long           nr_active;
        unsigned long           nr_inactive;
        int                     all_unreclaimable;
        unsigned long           pages_scanned;
        int                     temp_priority;
        int                     prev_priority;
        struct free_area        free_area[MAX_ORDER];
        wait_queue_head_t       *wait_table;
        unsigned long           wait_table_size;
        unsigned long           wait_table_bits;
        struct per_cpu_pageset  pageset[NR_CPUS];
        struct pglist_data      *zone_pgdat;
        struct page             *zone_mem_map;
        unsigned long           zone_start_pfn;
        char                    *name;
        unsigned long           spanned_pages;
        unsigned long           present_pages;
};
\end{lstlisting}

The kernel presents a set of functions to allocate memory: \textbf{kmalloc}, \textbf{vmalloc}. \textbf{kmalloc} is the most commonly invoked function which allocates physical memory.

Linux implements page caching by using physical pages in RAM to correspond to physical blocks on a disk.  Linux implements the \textbf{write-back} strategy for processing write operations.  Using this mechanism, the operating system makes write calls to the page-cache directly, only periodically writing these back to disk.

\section{FreeBSD and Linux}

	FreeBSD and Linux are very similar with respect to their memory management.  When a process is created, it shares the virtual address space of its parent process by default.  In general, pages are only copied from process when the child process writes to those pages.  This is known as \textit{copy-on-write} and is a memory optimization used by Windows Linux and FreeBSD to minimize page use.

Like Linux, FreeBSD uses an array of page structures which represent all the physical pages on the system.  In FreeBSD, these pages are represented by \textbf{struct vm\_page\_t}.  

FreeBSD differs slightly from Linux with respect to the specific implementation of the collection of \textbf{struct vm\_page\_t}.  FreeBSD pages can be in one of five states: wired, active, inactive, cache or free.

\code{}
struct vm_page {
	TAILQ_ENTRY(vm_page) listq;	/* pages in same object (O) */
	vm_object_t object;		/* which object am I in (O,P) */
	vm_pindex_t pindex;		/* offset into object (O,P) */
	vm_paddr_t phys_addr;		/* physical address of page */
	struct md_page md;		/* machine dependant stuff */
	u_int wire_count;		/* wired down maps refs (P) */
	volatile u_int busy_lock;	/* busy owners lock */
	uint16_t hold_count;		/* page hold count (P) */
	uint16_t flags;			/* page PG_* flags (P) */
	uint8_t aflags;			/* access is atomic */
	uint8_t oflags;			/* page VPO_* flags (O) */
	uint8_t	queue;			/* page queue index (P,Q) */
	int8_t psind;			/* pagesizes[] index (O) */
	int8_t segind;
	uint8_t	order;			/* index of the buddy queue */
	uint8_t pool;
	u_char	act_count;		/* page usage count (P) */
	/* NOTE that these must support one bit per DEV_BSIZE in a page */
	/* so, on normal X86 kernels, they must be at least 8 bits wide */
	vm_page_bits_t valid;		/* map of valid DEV_BSIZE chunks (O) */
	vm_page_bits_t dirty;		/* map of dirty DEV_BSIZE chunks (M) */
};
\endcode{}

FreeBSD keeps instances of the \textbf{struct vm\_page\_t} stored in a doubly-linked array.  For each possible page state, FreeBSD maintains a set of queues of varying sizes (based on the size of processor's L1/L2 Cache)

\section{Windows Memory Management}

Windows memory management has significant differences compared to FreeBSD and Linux.  Virtual memory is provided by a set of system services known as the Memory Manager.  These services include allocation and de-allocation of virtual memory, memory sharing between processes and others.

Due to the difference in which Windows views threads vs processes, it allocates virtual address mappings differently than FreeBSD and Linux.  When a process is allocated in 32bit Windows, the operating system allocates a default of 2GB of virtually mapped memory.

The \textit{Memory Manager} has two main tasks: translating virtual address spaces into physical memory and handling memory paging when the system is using more RAM than available.  The services of the \textit{Memory Manager} are provided through the Windows API via the  ... calls.

Windows also represents internal memory using pages which are similar to FreeBSD and Linux.  Windows pages have four states: free, reserved, committed, or shareable.

Unlike Linux, Windows supports only two page sizes...

Like Linux and FreeBSD, Windows uses copy-on-write technique to optimize memory usage.  If a process maps a section of read/write shared memory, the memory will only be copied when the process tries to modify it.
