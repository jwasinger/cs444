Processes and threads are two key constructs within modern multiprocessor systems.  In order to take advantage of processor resources, multiple threads of execution are created, and from their own point of view run simultaneosuly with other processes.  In order for processors to maintain this abstraction, processors schedule threads in a way that guarantees that CPU resources are dolled out most efficiently.

Windows, Linux and FreeBSD all share the concepts of threads, processes and CPU scheduling.  However, there are some important fundamental differences between Windows and FreeBSD/Linux that highlight key architectural differences between the two

\section{FreeBSD}

	Processes in FreeBSD are similar to Linux in many ways.  The relationship between processes and threads in Linux and FreeBSD systems are hierarchical.  When a system is booted, a special `init` process is started to complete system initialization.  All subsequent processes exist as children of the `init` process.  The system call `clone` is used to create a new process \cite{lkd}.

	In Linux processes are implemented in the kernel via a doubly-linked list known as the `task list`.  Each entry in the `task list` contains information about the state of a given process on the system \cite{lkd}.

\begin{lstlisting}
struct thread_info {
        struct task_struct    *task;
        struct exec_domain    *exec_domain;
        unsigned long         flags;
        unsigned long         status;
        __u32                 cpu;
        __s32                 preempt_count;
        mm_segment_t          addr_limit;
        struct restart_block  restart_block;
        unsigned long         previous_esp;
        __u8                  supervisor_stack[0];
};
\end{lstlisting}

	Both Linux and FreeBSD threads maintain a kernel stack.  This is so that Kernel operations on user-space process can be done without exposing sensitive Kernel information to the user-space.

	The implementation of processes and threads in FreeBSD differs from Linux in several important ways.  
	FreeBSD makes a clear distinction between threads and processes.  Threads must be spawned by a process.  In their lightest form, threads can share nearly all resources with their parent process (including PID) \cite{freebsd}.

	Both Linux and FreeBSD conform to the POSIX Threads standard.  This was probably chosen because  having multiple operating systems support a common interface for concurrency is mutually benefical to both FreeBSD and Linux.  This is because it is easier to write programs that can be cross-compiled between FreeBSD and Linux operating systems.

	The way Linux schedules thread execution is called the \textit{Completely Fair Scheduler}.  It gives each process a roughly equal proportion of the processor's time.  Unlike priority-based scheduling systems \textit{CFS}, decides whether a process should continue running based counting how long it has been running already.  In a time-slice based system, the processor would assign a process a countdown timer for the 'timeslice' that it can run for.  When that timer ticks to zero, the process is pre-empted.

	FreeBSD, unlike Linux uses a priority-based thread scheduling system.  In FreeBSD CPU resources are given to threads based on \textit{scheduling class} and \textit{scheduling priority}.  FreeBSD has two kernel scheduling classes and three user scheduling classes \cite{freebsd} :
		\begin{itemize}
			\item (0-47) bottom-half kernel (interrupt)
			\item (48-79) real-time user
			\item (80-119) top-half kernel
			\item (120-223) time-sharing user
			\item (224-255) idle user
		\end{itemize}

	The traditional FreeBSD thread scheduler uses a list of all running threads of execution.  Traversing the list once per second, the operating system analyzes thread performance and requirements and adjusts thread priorities accordingly.  This had the effect that, as the number of threads increased over time, traversing the list would become infeasible and represent a significant performance bottleneck.  To address this, the ULE thread scheduling algorithm was developed.

	In FreeBSD, threads in the bottom-half kernel class will always be given preference.  Tasks in this class are kernel-intterupt threads that need to be run as soon as possible (97).  Threads in the real-time and idle classes can be set by the application using the `rtprio` system call.

	Traditional FreeBSD thread scheduling is similar to Linux in its goal.  FreeBSD thread scheduling is implemented in the form of a round-robin allocated thread system through assignment of processor time slices to threads.  Like Linux, FreeBSD scheduling gives priority to interactive jobs by increasing the scheduling priority of threads that are blocked on I/O operations for 1 or more second (125-126).  
	I assume that this feature was implemented on both operating systems in order to make them more viable for use on desktop machines which typically run more interactive programs than traditional mainframe computers.

\section{Comparing Windows with Linux}
Windows, like FreeBSD implements a priority-based thread scheduling system.  Each thread is assigned a scheduling priority.  Threads of the same priority are run in a round-robin manner whereby each thread is given a fixed time slice to run for.  Threads in lower priority classes can be pre-empted at any time by threads in higher priority classes.

Unlike Linux-based operating systems which make no distinction between threads and processes, Windows defines a clear distinction between threads and processes.  Each process in windows is represented by an instance of an executive process (EPROCESS) structure.  Among other things, this structure also contains references to the threads allocated for any given process.  For each process that is executing a Win32 executable, there is a separate structure maintained in parallel by the Win32 subsystem process (CRSS) \cite{windows}.
	Windows introduces the concept of a 'protected process'.  An administrator has full control over the code running in any process on the system.  However, this elevated privilege could potentially be used to perform illegal activities (such as extracting movie/song information from movie/video player processes).  Protected processes exist alongside normal process.  However, they add significant constraints to access rights from other processes executing on the system.  Protected processes can only be created if the executable image file has been signed with a special Windows Media Certificate \cite{windows}.  This was probably implemented this way so that Windows would be more attractive as a commercial platform for software.  This way, companies building video/sound editing and playback programs would feel more comfortable deploying their software to Windows.  It is also likely that Microsoft gets to make some extra profit by charging to issue the special Windows Media Certificates.

As compared to Linux, the system by which the Windows Operating system manages 
freebsd:
	http://queue.acm.org/detail.cfm?id=1035622
