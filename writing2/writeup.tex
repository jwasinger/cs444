I/O is an important subject in the study of operating systems.  On a low level, operating system I/O functionality defines how a computer interracts with crucial peripheral devices such as disk drives, hard drives, keyboards and mice among many more.  Windows, Linux and FreeBSD all expose I/O operations that make it easier for applications to take advantage of computing resources without being aware of implementation quirks of underlying hardware.

Windows, FreeBSD and Linux implement a driver model to orchestrate access to computer peripheral devices.  However, the I/O models of the three operating systems are markedly distinct in important ways.

	Scatter-gather I/O is another supported feature of all three operating systems.  In this model, data from multiple buffers is written to a single data stream.  This is done atomically, whereby the system guarantees that all reads or writes have occured without interference from other processes. 

* Support of synchronous/asynchronous I/O *

Device I/O on FreeBSD and Linux is very similar.  Hardware devices are exposed as files mounted under the '/dev' directory.  Because UNIX likes to treat everything like a file, devices can be interracted with exactly as normal files can.  Devices are controlled by device drivers.  Device drivers can be compiled into the kernel directly or loaded at runtime using the 'insmod' and 'kld' commands on Linux and FreeBSD respectively.

In Linux systems, there are two types of devices: block and character devices.  These are also respectively referred to as structured and unstructured devices *.  Block devices allow random read/write access to fixed-size chunks of data.  Block device memory is typically cached by the kernel.  This makes block devices suitable for the implementation of file systems.

  This means that the time when data is written from cache to the actual device is not known.  For this reason, read data from block devices cannot be expected to be perfectly up-to-date with the contents of the physical device if recently written data has not yet been flushed from cache to disk.  Due to this flaw, FreeBSD no longer supports block devices *.

  Character devices are known as 'unstructured' because they are accessed as a stream of bytes instead of structure blocks.
  Character devices require that data is provided in a serial fashion and typically allow large I/O operations to take place.


	Linux device drivers *

	
FreeBSD: scatter-gather I/O 

Windows:

  The Windows implementation of I/O functionality has som key differences compared to Linux and FreeBSD.
  The Windows implementation of I/O functionality is almost completely different from an architectural point of view compared to the I/O models of FreeBSD and Linux. The Windows I/O subsytem has, at its core, the \textit{I/O Manager}.  This system defines a model for packet-driven I/O whereby, I/O requests are passed via a packet system to their correct drivers.
  
  When the \textit{I/o Manager} receives an I/O request, it generates an \textit{IRP (Interrupt Request Packet} which is then consumed by the driver that this \textit{IRP} pertains to.

  Key similarity to Linux: I/O subystem operates on virtual files only.  It is the responsibility of the driver to translate virtual file I/O requests into low-level operations on the specific hardware.

  Compared to FreeBSD and Linux, Windows has a much larger variety of driver types.
  
  Printer drivers  help translate graphical content into a printable form.  They typically forward print commands to the kernel space.  This is likely because there are many applications for which Windows needs to have tight control over what is being printed to prevent certain crimes such as counterfet money.

  (User mode driver framework?)

  Kernel Mode Drivers:
    File System Drivers translate virtual filesystem requests into media-specific commands (network storage, etc.).
	Drivers which handle peripheral devices usually fall under the management of the Plug and Play system.  The plug and play system allows the operating system to automatically recognize hardware and determine correct drivers for this hardware.  This is done using a system known as the \textit{PnP Manager}.
	The plug and play system falls under the \textit{Windows Driver Model (WDM)}.  Under this model,  there are three driver types.  \textit{Bus drivers} manage logical and physical buses.  \textit{Function drivers} are given resources by \textit{Bus drivers} by the \textit{PnP Manager}.  \texit{Function drivers expose an interface to the Operating System that abstracts away the details of the hardware they manage.

  Windows 

  *I/O model*
  *Plug and Play*

These key differences between Windows and Linux highlight the difference in design methodology.  Linux prefers to design components that are modular, and perform one thing well.  The developers of Windows have chosen to enhance the functionality of the operating system at the cost of increased complexity.

--------------
I/O Scheduling
--------------

Many devices contend for device resources.  In a multi-threaded environment, there are many programs that are constantly accessing storage devices.  In order to make sure that data is written and read as efficiently as possible, operating systems must schedule the individual operations correctly.  devices such as mechanically spinning-disk hard drives run inneficiently if the scheduled I/O operations are not ordered correctly.

Linux:
  In Linux, I/O scheduling is done on a per-disk bases

Windows:

 The Windows  I/O scheduling model sees I/O requests as priority based.  There are five I/O priorities: \textit{Critical, High, Normal, Low, Very Low}.  By default, I/O requests have priority of \textit{Normal}.  The Windows Memory Manager uses \textit{Critical} when it is flushing dirty memory from RAM to disk under low memory load.  All Windows background tasks *expand?* use \textit{Very Low} background priority.
 
 These priorities are divided internally into two types of I/O prioritization called \textit{strategies}: \textit{hierarchy prioritization} and \textit{idle prioritization}. \textit{Hierarchy prioritization} applies to all I/O \textit{priorities} except \textit{Very Low}. 

All \textit{Critical priority} I/O must be processed before any \textit{High priority} I/O.  All \textit{High priority} I/O must be processed before any \textit{Normal priority} I/O.  All \textit{Normal priority} I/O must be processed before any \textit{Low priority} I/O.  All \textit{Low priority} I/O is processed after any \textit{High priority} I/O.

When \textit{IRPs} are received, they are put onto different I/O queues based on their calculated priority.


freebsd
-------
https://www.freebsd.org/doc/en/books/arch-handbook/driverbasics-char.html
https://www.freebsd.org/doc/en/books/arch-handbook/driverbasics.html


linux
-----
http://free-electrons.com/doc/books/ldd3.pdf

windows
-------
https://www.microsoftpressstore.com/articles/article.aspx?p=2201309
https://msdn.microsoft.com/en-us/library/windows/hardware/ff548102(v=vs.85).aspx
