As has been shown in this brief overview, the Linux, FreeBSD and Windows operating systems have important differences in the details of their implementation and the features that they offer.  These differences highlight the target-users of the operating systems.  Windows, with a focus on the average consumer has put considerable resources into the development of the \textit{Windows Driver Model} and the \textit{Plug and Play} driver subsystem.  The ease by which the operating system handles peripheral devices is an important factor in its dominance in the personal computing market.  FreeBSD, while very similar in implementation and features to Linux, has some important differences that highlight the difference in design compare to Linux.  While Linux aims to make components of its operating system highly modular (potentially at the expense of flexibility and power), FreeBSD has implementated several features which fly in the face of this design goal.  An example of this is the priority-based thread scheduling system that FreeBSD uses.  This is somewhat similar to Windows and higlights the fact that the developers of FreeBSD chose to create a more fully-featured thread scheduling system at the expense of creating a more complex architecture.

The comparison here only higlights  several differences between FreeBSD, Windows and Linux within several important areas of operating system implementation
