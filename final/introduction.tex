The study of operating system implementation is a very important topic.  Although most computer professionals operate at a level of abstraction far higher than that of the operating system kernel, an understanding of the system pieces and the code that implements them can be key for giving an enhanced insight into the operation of computer systems.

In this class.  We have focused our studies on the implementation of the Linux kernel.  By studying vital subsystems such as memory management, process thread creation and lifetime, the details of implementing device drivers, understanding the practical applications of kernel cryptography we have gained an insight and appreciation into the inner mechanics of the Kernel.  However, this knowledge would be less useful without having a more broad overview of the design of Linux as it compares to other industry standard operating systems.

For the purpose of this study, we have elected to compare and contrast the implementation of the Linux kernel to that of FreeBSD and Windows operating systems.  By understanding the design differences in these industry standard operating systems, we gain a valuable insight into the minds of the developers and design methodologies that they chose.  Feature tradeoffs between the three operating systems highlight the differences in the ecosystems that they have created and perpetuate.
